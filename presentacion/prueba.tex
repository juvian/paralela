\documentclass[compress]{beamer}
\newcommand{\quotes}[1]{``#1''}

\mode<presentation>
{
  \usetheme{Warsaw}
  % or ...
	\useoutertheme{infolines}
    \usecolortheme{spruce}
  \setbeamercovered{transparent}
  % or whatever (possibly just delete it)
}

\usepackage{ upgreek }
\usepackage{verbatim} 
\usepackage{listings}
\usepackage{tikz}
\usepackage{algorithm,algpseudocode}
\usetikzlibrary{arrows}
\usetikzlibrary{shapes}
\tikzstyle{block}=[draw opacity=0.7,line width=1.4cm]

\lstloadlanguages{C++}
\lstnewenvironment{code}
	{%\lstset{	numbers=none, frame=lines, basicstyle=\small\ttfamily, }%
	 \csname lst@SetFirstLabel\endcsname}
	{\csname lst@SaveFirstLabel\endcsname}
\lstset{% general command to set parameter(s)
	language=Python, basicstyle=\footnotesize\ttfamily, keywordstyle=\slshape,
	emph=[1]{tipo,usa}, emphstyle={[1]\sffamily\bfseries},
	morekeywords={tint,forn,forsn},
	basewidth={0.47em,0.40em},
	columns=fixed, fontadjust, resetmargins, xrightmargin=5pt, xleftmargin=15pt,
	flexiblecolumns=false, tabsize=2, breaklines,	breakatwhitespace=false, extendedchars=true,
	numbers=left, numberstyle=\tiny, stepnumber=1, numbersep=9pt,
	frame=l, framesep=3pt, escapeinside={(*}{*)},
}

\usepackage{multicol}

\usepackage[spanish]{babel}
% or whatever

\usepackage[utf8]{inputenc}
% or whatever

\usepackage{times}
\usepackage[T1]{fontenc}
% Or whatever. Note that the encoding and the font should match. If T1
% does not look nice, try deleting the line with the fontenc.
\usepackage{enumitem}
\usepackage{booktabs}


\title[Programación Paralela Basada en Patrones] % (optional, use only with long paper titles)
{Números primos en un rango determinado}

\author[Bekier Lucas, Del Gobbo Julián] % (optional, use only with lots of authors)
{Bekier Lucas, Del Gobbo Julián}
% - Give the names in the same order as the appear in the paper.
% - Use the \inst{?} command only if the authors have different
%   affiliation.
\institute[UBA FCEN] % (optional, but mostly needed)
{
  Facultad de Ciencias Exactas y Naturales\\
  Universidad de Buenos Aires
}


% Ac� se puede insertar el logo de la UBA
% \pgfdeclareimage[height=0.5cm]{university-logo}{university-logo-filename}
% \logo{\pgfuseimage{university-logo}}


% Delete this, if you do not want the table of contents to pop up at
% the beginning of each subsection:
\AtBeginSubsection[]
{

  \begin{frame}<beamer>{Contenidos}
    \tableofcontents[currentsection,currentsubsection]
  \end{frame}
	
}


% If you wish to uncover everything in a step-wise fashion, uncomment
% the following command: 

%\beamerdefaultoverlayspecification{<+->}


\begin{document}
\pgfdeclarelayer{background}
\pgfsetlayers{background,main}
\begin{frame}
  \titlepage
\end{frame}

\begin{frame}{Contenidos}
  \tableofcontents
  % You might wish to add the option [pausesections]
\end{frame}



\section{Introducci\'on al problema}

\begin{frame}{Introducci\'on al problema}

  \begin{itemize}
    \setlength\itemsep{1em}
    \item<1-> Un número primo es aquel que es divisible sólo por sí mismo y el 1.
    \item<2-> Existen muchas formas de calcular si un número en particular es primo.
  	\begin{itemize}
  	\setlength\itemsep{1em}
  		\item<3-> Dividiendo el número por todos los anteriores hasta la raiz cuadrada y ver el resto en cada una.
      \item<4-> Usar algoritmos probabilísticos como Miller Rabin que utilizan conceptos matemáticos más avanzados.
      \item<5-> Calcular la criba de Eratóstenes hasta un número superior y consultar si lo es o no.
  	\end{itemize} 
  \end{itemize}
\end{frame}


\begin{frame}{Introducci\'on al problema}
  \begin{itemize}
    \setlength\itemsep{1em}
    \item<1-> La criba conviene usarla cuando se desea calcular todos los primos desde 1 hasta N.
    \item<2-> ?` Pero qué pasa si queremos calcular los primos entre X e Z ?
    \item<3-> \textbf{La criba común no es la mejor elección.}
  \end{itemize}
\end{frame}

\begin{frame}{Introducci\'on al problema}
  \begin{enumerate}
    \setlength\itemsep{1em}
    \item<1-> El algoritmo calcula todos los primos entre 1 e Z.
    \item<2-> Desaprovechamos la propiedad matemática que dice que necesitamos solamente los primos hasta la raiz cuadrada de un número para determinar su primalidad.
    \item<3-> Es un algoritmo completamente secuencial.
  \end{enumerate}
\end{frame}

\section{Encarando el problema de forma paralela}

\begin{frame}{Como mejorar la performance?}
  \begin{center}
    \includegraphics[scale=0.75]{imagenes/parallelism.jpg}
  \end{center}
\end{frame}

\begin{frame}[fragile]{Paralelismo Versión 1}
  \begin{itemize}
    \setlength\itemsep{1em}
    \item<1-> Proponemos el siguiente algoritmo naive que cuenta la cantidad de primos entre un número X e Z con un algoritmo naive.
    \item<2->
     \begin{lstlisting}
        (*\bfseries dep:*) D(j) -> Y(j, i)
        (*\bfseries dep:*) Y(j,i) -> L(i)
        (*\bfseries dep:*) L(i) -> C
        (*\bfseries forall:*)(* X $\leq$ i $\leq$ Z *) 
          (*\bfseries par*)
            (*\bfseries forall:*) (*2 $\leq$ j $\leq$ $\sqrt{i}$*)
              D: d = j  
            (*\bfseries forall:*) d
              Y: y = (*\textit{divides}*)(d, i)
            L: l = !(*\textit{any}*)(y)
        C: (*\textit{count}*)(l)
    \end{lstlisting}  
    \item<3-> Como optimización se puede contemplar únicamente números impares entre los candidatos a divisores.        
  \end{itemize}
\end{frame}


\begin{frame}[fragile]{Paralelismo Versión 2}
  \begin{itemize}
    \setlength\itemsep{1em}
    \item<1-> El siguiente algoritmo busca usar únicamente los primos hasta $\sqrt{Z}$ como posibles divisores.
    \item<2->
     \begin{lstlisting}
        seq
          P: p = (*\textit{primes}(1, $\sqrt{Z}$)*)
          (*\bfseries forall:*)(* X $\leq$ i $\leq$ Z *) 
            (*\bfseries forall:*) p
              Y: y = (*\textit{divides}*)(p, i)
            L: l = !(*\textit{any}*)(y)
          C: (*\textit{count}*)(l)
    \end{lstlisting}
    \item<3-> En este caso, \textit{primes} es una función legacy que calcula los primos hasta la raiz, que si es lo suficientemente rápida genera mucha menor carga en la siguiente parte del algoritmo.          
  \end{itemize}
\end{frame}

\begin{frame}[fragile]{Paralelismo Versión 3}
  \begin{itemize}
    \setlength\itemsep{1em}
    \item<1-> En lugar de para cada número tratar de ver si es primo, hacer como la criba y marcar para cada número entre X e Y si uno lo divide.
    \item<2-> Uso de chunking para ganar velocidad.
    \item<3-> En pseudo FXML:
      \begin{lstlisting}
          P =(*\textit{primes}(1, $\sqrt{Z}$)*)
          Parto el espacio en ((Z - X) / chunksize) chunks de tamano chunksize
          (*\bfseries forall:*) chunk
            (*\bfseries forall:*) P
              Y: y = (*\textit{mark\_multiples\_in}*)(chunk, P)
            T: t = (*\textit{sum}*)(y)
          C = (*\textit{sum}*)(t)
      \end{lstlisting}
  \end{itemize}
\end{frame}

\begin{frame}{Grandes resultados}
  \begin{center}
  {4 cores, 8 logical processors}
  \\
  \includegraphics[width=1.15\textwidth]{imagenes/todoJulian.png}%
  \end{center}
\end{frame}

\begin{frame}{Grandes resultados}
  \begin{center}
  {4 cores, 8 logical processors}
  \\
  \includegraphics[width=1.15\textwidth]{imagenes/principioJulian.png}%
  \end{center}
\end{frame}

\begin{frame}{El overhead mata}
  \begin{center}
  {2 cores, 2 logical processors}
  \\
  \includegraphics[width=1.15\textwidth]{imagenes/todoLucas.png}%
  \end{center}
\end{frame}

\begin{frame}{El overhead mata}
  \begin{center}
  {2 cores, 2 logical processors}
  \\
  \includegraphics[width=1.15\textwidth]{imagenes/principioLucas.png}%
  \end{center}
\end{frame}

\begin{frame}
  \begin{center}

  {\Huge ?` Preguntas ?}
  \\
  \includegraphics[scale=0.235]{imagenes/preguntas.png}

  \end{center}
\end{frame}

\end{document}
